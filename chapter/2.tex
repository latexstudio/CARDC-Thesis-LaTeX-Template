\chapter{模板介绍}
\section{包含的主要文件}
\begin{itemize}
\itemsep=0pt \parskip=0pt
\item \textbf{main.tex:}位于根目录下,是本模板的主要文件。
\item \textbf{bib文件夹:}包含的是参考文献的bib数据库。%
\item \textbf{CARDCDef文件夹:}是本模板的格式设置文件,其中CARDC.cls是
本模板格式的主要设置文件,CARDCTitlePage.sty文件定义了本模板的中英文扉
页和版权声明页,YYBib.sty文件定义了参考文献格式。一般情况下,这三个文件请勿修改。
\item \textbf{chapter文件夹:}是本模板章节内容放置处。
\item \textbf{pic文件夹:}放置了本模板用的图片。
\item \textbf{Script文件夹:}编译调用的脚本文件。
\end{itemize}

\subsection{main.tex文件}
您毕业论文的各项信息请根据注释部分的提示填写即可。

\subsection{ref.bib文件}
在bib文件夹下的ref.bib文件是本模板使用的参考文献数据库。该数据类型可以%
使用其他文献管理软件生成,见本模板附带的图文说明文档《第一次使用LaTeX?读我》。%

\subsection{MyDef.sty文件}
如果您在使用该模板的过程中,需要用到该模板没有导入的宏包或者增加自定义命令,%
可以在导言区增加,或者在CARDCDef文件夹下的MyDef.sty文件中添加,\textbf{本人推荐您%
在MyDef.sty文件中添加,方便统一管理。}

\subsection{Compile.bat}
Windows下的批处理文件,实现一键编译,在文件里面可以切换参考文献后台处理程序,%
如果使用切换至Biber,\textbf{请以管理员权限运行该脚本}。

\subsection{Clean.bat}
Windows下清理临时文件的批处理文件。

\subsection{Makefile}
在Linux环境下(Linux系统或者Cygwin环境),可以使用make工具编译。

\subsection{章节内容}
章节内容包含在chapter文件夹中,已有的是
\begin{itemize}
\itemsep=0pt \parskip=0pt
\item \textbf{abstract.tex:}是中英文摘要。
\item \textbf{appendix.tex:}是附录内容,如果没有附录内容,在main.tex中将该部分%
    导入注释掉。%
\item \textbf{thanks.tex:}是致谢部分。
\item \textbf{1.tex,2.tex:}是本说明文档的第一、二章内容。
\end{itemize}

如果您增加新的章节,请使用UTF-8编码格式,并在main.tex中使用include命令导入即可。%
关于如何将编码格式转为UTF-8,见本模板附带的图文说明文档《第一次使用LaTeX?读我》。%
本人推荐您直接复制chapter目录下的.tex文件,改为需要的名字并导入,然后在里头直接%
书写内容即可。
